\documentclass[11pt]{beamer}
%\usetheme{Berkeley}
\usetheme{Warsaw}
%\usetheme{AnnArbor}
%\usetheme{Berlin}
%\usetheme{Antibes}
%\usetheme{Bergen}
%\usetheme{Szeged}
%\usetheme{CambridgeUS}
%\usetheme{Malmoe}
\usepackage[utf8]{inputenc}
\usepackage[english]{babel}
\usepackage{amsmath}
\usepackage{amsfonts}
\usepackage{amssymb}
\usepackage{xcolor}
\usepackage{graphicx}

\author{  B.EL FALLAH\\ {\color{blue}{ Coordonateur:}} Dr.ELHAJJAJI Otmane }
\title{Méthode d'Euler en language Fortran90}
%\setbeamercovered{transparent} 
%\setbeamertemplate{navigation symbols}{} 
\logo{\includegraphics[scale=0.3]{fst.png}} 
\institute{UAE, Faculté des sciences de Tétouan} 
%\date{} 
%\subject{} 

\begin{document}

\begin{frame}
\frametitle{TRAVAUX PRATIQUES DE METHODES NUMERIQUES}
\titlepage
\end{frame}

\begin{frame}
\frametitle{Planing}
\tableofcontents
\end{frame}

\begin{frame}%{Principe de la méthode}
\frametitle{Description de la méthode}
\section{Principe de la méthode}
\begin{block}


La méthode d'euler consiste à chercher une solution approchée de :
{\color{blue}{  $$\dfrac{dy}{dx}=f(x,y)$$ }} avec { \color{blue}{ $y_{0}=y(x_{0})$}}
sur l'intervalle $ I=[x_{0},x_{N}] $\\
le pas d'intégration est :{\color{blue}{  $$ h=\frac{x_{N}-x_{0}}{N}$$\\}}
\end{block}
\end{frame}
\begin{frame}
\frametitle{Description de la méthode}
\begin{minipage}[r]{0.48\textwidth} \hfill

\begin{block}


{\color{red}{
Géométriquement}} : Remplacer la courbe sur $$ [x_{i} , x_{i}+h] $$ par la tangente.
Si $$ M_{N}(x_{n} , y_{n})$$ est le point courant de la courbe "solution", le nouveau point :\\
 $$ M_{N+1}(x_{N} +h , y_{N}+ h\times f(x_{N} , y_{N})) $$ .
\end{block}
\end{minipage}
\begin{minipage}[l]{0.45\textwidth} 
\includegraphics[scale=0.4]{courbe.png}
\end{minipage}
\end{frame}
\begin{frame}
\section{Application de la méthode}
\subsection{Résolution analytique}
\frametitle{Résolution analytique}
\begin{block}


la résolution de l'EDO suivante :
{\color{blue}{  $$ \dfrac{dy}{dx}=x+y $$ }}
avec { \color{blue}{ $$ y_{0}=1 $$}}
est obtenue analytiquement comme suit :
$$ y(x)=2e^{x}-x-1 $$.
\end{block}
\end{frame}
\begin{frame}
\subsection{Résolution numerique sur machine}
\subsubsection{Algorithme}
\frametitle{Algorithme}
\includegraphics[scale=0.49]{alg.png}
\end{frame}
\begin{frame}
\subsubsection{Programme en language gFortran}
\frametitle{Code source du programme en gFortran}
\begin{block}


\includegraphics[scale=0.3]{prog.png}
\end{block}


\end{frame}

\begin{frame}
\subsubsection{Resultat du programe}
\frametitle{Resultat du programe}
\includegraphics[scale=0.44]{res.png}

\end{frame}
\begin{frame}
\subsubsection{Representation graphique}
\frametitle{Representation graphique}
\includegraphics[scale=0.6]{gra.png}
\end{frame}

\end{document}